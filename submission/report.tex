\documentclass[conference]{IEEEtran}
\IEEEoverridecommandlockouts
% The preceding line is only needed to identify funding in the first footnote. If that is unneeded, please comment it out.
\usepackage{cite}
\usepackage{url}
\usepackage{amsmath,amssymb,amsfonts}
\usepackage{algorithmic}
\usepackage{graphicx}
\usepackage{textcomp}
\usepackage{xcolor}

\def\BibTeX{{\rm B\kern-.05em{\sc i\kern-.025em b}\kern-.08em
    T\kern-.1667em\lower.7ex\hbox{E}\kern-.125emX}}
\begin{document}
\bibliographystyle{IEEEtran}
\title{\includegraphics[width=5cm]{images/ust.png}\linebreak DreamHarmony: Unveiling Sleep Patterns through Lifestyle Modeling and Predictive Analytics
}

\author{\IEEEauthorblockN{Hoang Q. Nguyen\IEEEauthorrefmark{1}\IEEEauthorrefmark{3}, Md Mahmudul Hassan\IEEEauthorrefmark{1}\IEEEauthorrefmark{3}, Solomon Rukundo\IEEEauthorrefmark{2}\IEEEauthorrefmark{3}}
    \IEEEauthorblockA{\IEEEauthorrefmark{1} Korea Institute of Science and Technology}
    \IEEEauthorblockA{\IEEEauthorrefmark{2} Korea Research Institute of Standards and Science}
    \IEEEauthorblockA{\IEEEauthorrefmark{3} Korea National University of Science and Technology (UST)}
}

\maketitle
\thispagestyle{plain} % disable this for paper publication
\pagestyle{plain} % disable this for paper publication

\begin{abstract}
    This study investigates how daily habits like exercise and technology use are related to sleep quality. After careful data preparation, we used statistical methods, including Pearson\cite{pearson} and Spearman correlations\cite{spearman}, to identify which lifestyle factors are most connected to good or poor sleep, in terms of duration and quality. We then built various mathematical models, such as linear regression and decision trees, to predict sleep quality based on these factors. Our results show that some daily activities have a strong link to how well people sleep, and our models can explain a significant part of this link. These findings could help guide recommendations for improving sleep based on personal habits.
\end{abstract}

\begin{IEEEkeywords}
    big data, sleep, health, lifestyle, machine learning, support vector machine
\end{IEEEkeywords}

\section{Introduction}
Our study explored the complex relationship between lifestyle factors and sleep quality through a structured survey capturing various daily activities and sleep metrics. This comprehensive data collection focused on routines, technological use, and exercise habits. The data underwent extensive cleaning for analytical robustness, and an initial exploratory analysis using a correlation matrix heatmap identified potential variable relationships, setting the stage for hypothesis development.

We then tested these hypotheses using statistical methods such as Spearman's and Pearson's correlations for understanding monotonic and linear relationships, and the Chi-squared test for categorical data associations. The results guided the selection of key variables for developing machine learning models, including regression analyses, decision trees, and support vector machines, which were critically assessed for their predictive accuracy in the context of sleep quality.

Our contributions through this research are multifaceted. Initially, we developed a foundational survey engineered to assess sleep and lifestyle variables. Subsequently, we discerned pivotal factors that influence sleep quality by leveraging statistical analysis and hypothesis testing. Furthermore, we harnessed machine learning algorithms to construct models based on these factors. Our study establishes a framework for future exploration, highlighting the necessity for continuous data accumulation to refine and augment model efficacy. We invite researchers and the public alike to contribute to our ongoing work (available on GitHub\footnotemark) through feedback and new applications of our findings.
\footnotetext{\url{https://github.com/johnkimtech/sleep}}



\section{Related Work}
A large Japanese cross-sectional study (n = 30,000) found that 28\% slept less than 6 hours per night and 65\% slept less than 7 hours per night despite 80\% claiming adequate sleep \cite{lifestyleinfluence}. Logistic regression identified short sleep duration (\textless 6 hours) as associated with being female, younger age, urban living, unemployment, poor health, lack of exercise, and irregular eating \cite{lifestyleinfluence}. A review of Asian literature linked insufficient sleep to high stress, high BMI/obesity, and depression \cite{immune}. A study of Lithuanian university students (n = 405) found poor sleep quality, especially among medical students, to be associated with high academic demands, anxiety, limited leisure time, and academic dissatisfaction \cite{lithua}. These findings suggest that interventions targeting stress, leisure time, exercise, urban planning, mental health, and other lifestyle factors could potentially improve sleep \cite{lifestyleinfluence,lithua}. However, existing research lacks exploration of newer lifestyle factors, such as the impact of technology and social media on sleep habits. Future research is needed to investigate how emerging habits like nighttime electronic device use, pre-sleep social media engagement, and internet overuse might contribute to sleep disturbances.

\section{Data Acquition \& Preprocessing}
The DreamHarmony Sleep Survey was an integral part of our study, designed to explore the multifaceted relationship between lifestyle habits and sleep quality. To capture a diverse set of responses, the survey was made available in multiple languages including Korean, English, Vietnamese, and Bengali.

\subsection{Data collection through online survey}
The DreamHarmony Sleep Survey was meticulously designed to delve into the myriad factors affecting sleep quality and patterns. This comprehensive survey aimed to unravel the complex interplay between lifestyle habits and sleep, potentially leading to enhancements in sleep quality and overall life satisfaction.
\subsubsection*{Survey Structure and Distribution}

The survey was structured into distinct sections: demographics, lifestyle, and sleep habits. It was distributed via Google Forms\cite{dhsurvey} through the authors' social networks, encompassing friends, colleagues, and family members. While this method facilitated rapid data collection, we acknowledge the potential for bias and sample imbalance due to the nature of the distribution channels.

\subsubsection*{Demographics}

Participants were queried about their age, gender, education level, and occupation. This demographic information was crucial to understanding the contextual background of each respondent.

\subsubsection*{Lifestyle and Sleep Habits}

The lifestyle section encompassed questions regarding exercise frequency, device usage, and screen time before sleep. In the sleep habits section, we focused on gathering detailed information about participants' sleep patterns. Key questions included:
\begin{itemize}
    \item\textbf{Bedtime and Wake-up Time}: These questions helped calculate the actual duration of nighttime sleep, a vital metric for assessing sleep quality.
    \item\textbf{Sleep Quality}: Respondents rated their overall sleep quality, providing us with a subjective assessment of their sleep experience.
    \item\textbf{Sleep Onset}: We inquired about the time it typically takes for respondents to fall asleep, aiming to understand sleep onset difficulties.
    \item\textbf{Sleep Medication}: This question ascertained whether participants use any medication to aid their sleep, which is indicative of sleep quality issues.
    \item\textbf{Sleep Disturbances}: The frequency of disturbances such as waking up during the night or experiencing restless sleep was also captured.
\end{itemize}
\subsubsection*{Contributions}
Utilizing a survey of 108 participants, this study, though limited by distribution methods, leverages statistical and machine learning analyses to offer a nuanced understanding of sleep dynamics and lay the groundwork for future research.\subsection{Preprocessing: Translation, Merging, and Cleanup}
\subsubsection*{Translation}
To harness the collected data effectively in the subsequent stages of analysis, it was imperative to amalgamate the survey responses from all language versions into a single dataframe. This posed significant challenges, as the survey questions and responses were presented in various languages, and even the English version contained verbose questions and answers. To address these issues, we:

\begin{enumerate}
    \item Constructed a JSON file containing a translation mapping from the original column headers to abbreviated column headers in English.
    \item Utilized the \texttt{rename} function from the \texttt{pandas} library to update the dataframe with the new headers \cite{dfrename}.
\end{enumerate}

For the translation of cell values, a similar approach was employed, where JSON files served as the translation mapping. These mappings were then applied to the dataframe using the \texttt{map} function to convert the cell values into concise English terms \cite{dfmap}.

\subsubsection*{Merging \& Cleanup}
Once translations for both headers and cell values were standardized, we employed the \texttt{concat} function from \texttt{pandas} to stack the individual dataframes from each language version into a unified dataframe \cite{pdconcat}. However, this merged dataset contained several inconsistencies, including:

\begin{itemize}
    \item Heights recorded as less than 100cm, primarily in the Bengali version of the survey, where respondents preferred inches to centimeters. For these cases, heights were converted from inches to centimeters.
    \item Respondents' confusion between 24-hour and 12-hour time formats, resulting in miscalculations of sleep duration, with some reported durations nearing 20 hours per day. We addressed this by filtering out erroneous values and correcting the time format.
\end{itemize}
Furthermore, actual night sleep duration was calculated from the 'Bedtime' and 'Wake-up Time' responses. For respondents providing height and weight, Body Mass Index (BMI) was computed, offering additional metrics pivotal to the analysis. Following these corrections, the dataset was ready for subsequent stages of examination.
\section{Analysis}
\subsubsection*{Descriptive statistics}
\begin{table}[htbp]
    \centering
    \caption{Descriptive statistics of the sleep study data}
    \label{tab:sleep_data1}
    \begin{tabular}{lrrrrr}
        \hline
              & Height (cm) & Weight (kg) & BMI   & Sleep Quality & Night Sleep\footnotemark \\ \hline
        count & 83          & 92          & 80    & 108           & 105                      \\
        mean  & 165.31      & 67.42       & 24.55 & 3.44          & 7.04                     \\
        std   & 8.32        & 12.80       & 4.25  & 0.82          & 1.37                     \\
        min   & 150.00      & 43.00       & 18.52 & 2.00          & 1.67                     \\ \hline
        25\%  & 160.00      & 59.80       & 22.02 & 3.00          & 6.50                     \\
        50\%  & 167.00      & 68.00       & 24.69 & 3.00          & 7.00                     \\
        75\%  & 171.00      & 75.00       & 27.31 & 4.00          & 8.00                     \\
        max   & 185.00      & 100.00      & 35.85 & 5.00          & 9.75                     \\
        \hline
    \end{tabular}
\end{table}
\footnotetext{Calculated night time sleep based on Bedtime and Wake-up time}

\begin{table*}
    \centering
    \caption{Descriptive statistics of sleep study participants}
    \label{tab:sleep_data2}
    \begin{tabular}{lrrrrr}
        \hline
               & Age Group              & Gender                             & Education Level    & Occupation       & Exercise (days/week) \\ \hline
        count  & 108                    & 108                                & 108                & 108              & 108                  \\
        unique & 5                      & 3                                  & 4                  & 7                & 4                    \\
        top    & 25-34                  & Male                               & Master's           & Student          & 1-2 Days             \\
        freq   & 72                     & 67                                 & 47                 & 47               & 43                   \\ \hline
               & Device Usage (hrs/day) & Screen Time Before Sleep (hrs/min) & Sleep Onset (min)  & Bedtime          & Wake-up Time         \\ \hline
        count  & 108                    & 108                                & 108                & 104              & 108                  \\
        unique & 4                      & 4                                  & 4                  & 18               & 20                   \\
        top    & 7+ Hours               & 30-60mins                          & 15-30mins          & 23:00            & 07:00                \\
        freq   & 43                     & 45                                 & 55                 & 24               & 18                   \\ \hline

               & Nap Duration (min)     & Sleep Duration (hrs/24hr)          & Sleep Disturbances & Sleep Medication & Language             \\ \hline
        count  & 108                    & 107                                & 108                & 108              & 108                  \\
        unique & 5                      & 3                                  & 5                  & 2                & 4                    \\
        top    & No Nap                 & 6hrs+                              & Rarely             & No               & English              \\
        freq   & 61                     & 64                                 & 48                 & 48               & 68                   \\ \hline
    \end{tabular}
\end{table*}


As shown in Table~\ref{tab:sleep_data1} and \ref{tab:sleep_data2}:
\begin{itemize}
    \item \textbf{Sleep quality}: Respondents rated their sleep quality around 3 on a 5-point scale, indicating moderate quality.
    \item \textbf{BMI}: The average Body Mass Index (BMI) is around 23.55, ranging from 16.5 to 39.4.
    \item \textbf{Night sleep duration}: The average night sleep duration is around 7 hours, with a wide range of 1.67 to 9.75 hours.\item \textbf{Age Group}: The most common age group among respondents is 25-34.
    \item \textbf{Gender}: A slightly higher number of male respondents compared to females.
    \item \textbf{Education Level}: The majority of respondents have a Master's degree.
    \item \textbf{Occupation}: Many respondents are students.
    \item \textbf{Exercise Days/Week}: '1-2 Days' is the most common response for exercise frequency.
    \item \textbf{Device Usage (hrs/day)}: A large portion of respondents use devices for '7+ Hours' per day.
    \item \textbf{Screen Time Before Sleep}: '30-60 Minutes' is the most common duration for screen time before sleep.
    \item \textbf{Sleep Disturbances}: 'Rarely' is the most frequent response, indicating that most respondents rarely experience sleep disturbances.
    \item \textbf{Sleep Medication}: The majority of respondents do not use sleep medication.
    \item \textbf{Language}: English is the most common language among respondents.

\end{itemize}

\subsubsection*{Distribution Analysis}
The analysis of sleep quality and night sleep duration distributions, as depicted in Figures \ref{fig:distsleep} and \ref{fig:distsleepwm}, reveals the following insights:

\begin{itemize}
    \item \textbf{Sleep Quality Distribution:}
          The majority of the respondents report their sleep quality within the range of 2 to 4, with a concentration of responses at a sleep quality score of 3. Notably, a smaller proportion of participants indicate experiencing excellent sleep quality, as reflected by the fewer scores of 5.

    \item \textbf{Night Sleep Duration Distribution:}
          Observations from the histogram suggest a normal distribution of sleep duration, predominantly centered around the 7-hour mark. This finding is in line with standard sleep recommendations. Notably, the occurrences of extremely short (\textless 5 hours) or long (\textgreater 9 hours) sleep durations are relatively infrequent.

    \item \textbf{Gender-Based Sleep Differences:}
          The boxplots in Figure \ref{fig:distsleepwm} indicate a striking similarity in sleep quality between male and female participants. Additionally, the comparison of night sleep duration across genders reveals closely aligned distributions, with minor variations primarily in the lower and upper quartiles.
\end{itemize}

These observations contribute to a nuanced understanding of the distribution patterns of sleep quality and duration among the survey participants, highlighting subtle differences and commonalities in sleep experiences.

\begin{figure}[ht]
    \centering
    \includegraphics[width=8cm]{images/distsleep.png}
    \caption{Distribution of Sleep Quality and Duration}
    \label{fig:distsleep}
\end{figure}
\begin{figure}[ht]
    \centering
    \includegraphics[width=8cm]{images/distsleepwm.png}
    \caption{Distribution of Sleep Quality and Duration across Gender}
    \label{fig:distsleepwm}
\end{figure}


\subsubsection*{Correlation analysis}
The heatmap in Figure \ref{fig:corrheat} illustrates these following correlations:
\begin{itemize}

    \item \textbf{Sleep Quality:} Strong negative correlation with Sleep Disturbances (-0.55), indicating better sleep quality is associated with fewer disturbances.
          Moderate negative correlation with Sleep Onset Time (-0.32), suggesting that quicker sleep onset is associated with better sleep quality.

    \item \textbf{BMI:} Slight negative correlation with Calculated Night Sleep Duration (-0.19), suggesting that higher BMI might be slightly associated with shorter sleep duration, although the relationship is weak.
          Moderate negative correlation with Device Usage (-0.28), indicating that higher BMI is associated with less device usage.

    \item \textbf{Calculated Night Sleep Duration:} Negative correlation with Age Group (-0.23), indicating that older age groups might have shorter sleep duration.
          All other correlations with Calculated Night Sleep Duration are weak.

    \item \textbf{Age Group:} Moderate positive correlation with BMI (0.31), suggesting that higher BMI values are more prevalent in older age groups.

    \item \textbf{Sleep Onset Time:} No significant correlations with other variables, aside from the moderate negative correlation with Sleep Quality.

    \item \textbf{Nap Duration:} Weak correlations with all other variables.

    \item \textbf{Exercise Days/Week:} Slight positive correlation with Calculated Night Sleep Duration (0.22), implying that more exercise might be related to slightly longer sleep duration.
          Weak correlations with all other variables.

    \item \textbf{Sleep Disturbances:} Aside from the strong negative correlation with Sleep Quality, Sleep Disturbances show weak correlations with other variables.

    \item \textbf{Device Usage (hrs/day):} Moderate negative correlation with BMI (-0.28), as previously mentioned.
          Weak correlations with all other variables.
          This heatmap indicates that while some variables are correlated, most relationships are weak. The strongest observed relationships involve sleep quality, particularly its negative correlation with sleep disturbances and sleep onset time. This suggests that variables affecting the quality of sleep have a more significant impact on sleep disturbances and the time it takes to fall asleep. The correlations involving BMI, age group, and device usage suggest demographic and behavioral patterns but are not strong enough to imply causation.
\end{itemize}
\begin{figure}[ht]
    \centering
    \includegraphics[width=8cm]{images/corrheat.png}
    \caption{Correlation Heatmap between variables}
    \label{fig:corrheat}
\end{figure}
\section{Hypothesis Testing}
\subsection*{Overview}

The observed correlations from the data analysis and visualizations suggest several hypotheses that could be explored through further analysis:
\begin{itemize}

    \item \textit{Impact of Sleep Disturbances on Quality:} Given the strong negative correlation between sleep disturbances and quality, we can hypothesize that increased sleep disturbances are likely to negatively impact the quality of sleep.

    \item \textit{Age in Relation to Sleep Duration:} The negative correlation between age and calculated night sleep duration leads to the hypothesis that sleep duration may decrease with age.

    \item \textit{Relationship Between Sleep Onset Time and Quality:} The moderate negative correlation observed between sleep onset time and quality suggests that a longer time to fall asleep might be associated with poorer sleep quality.

    \item \textit{Influence of Exercise on Sleep Duration and Quality:} The slight positive correlation between exercise days per week and sleep duration hints at a potential hypothesis that increased physical activity could contribute to longer and possibly better quality sleep.

    \item \textit{Nap Duration's Effect on Nighttime Sleep Duration:} Although the correlation is weak, we could investigate whether the duration of naps has any effect on the duration of nighttime sleep.
\end{itemize}
\subsection*{Hypothesis I - Increased Sleep Disturbances Negatively Impact Sleep Quality}
\textit{Null Hypothesis (\(H_0\))}: The level of Sleep Disturbances has no impact on Sleep Quality.

\textit{Alternative Hypothesis (\(H_1\))}: The level of Sleep Disturbances negatively impacts Sleep Quality.

To investigate the relationship between sleep disturbances and sleep quality, we utilized Spearman's rank correlation due to its ability to measure monotonic relationships without assuming a linear relationship between variables, which is suitable for ordinal data. Additionally, the Chi-squared test was employed to evaluate the independence of categorical variables.

\begin{table}[ht]
    \centering
    \caption{Statistical Tests for Sleep Disturbances and Sleep Quality}
    \label{tab:hypothesis1}
    \begin{tabular}{|c|c|c|c|}
        \hline
        \textbf{Test Type}  & \textbf{Statistic} & \textbf{P-value}        & \textbf{Conclusion}            \\
        \hline
        Spearman's \(\rho\) & -0.453             & \(4.2 \times 10^{-7}\)  & Reject \(H_0\), accept \(H_1\)
        \\
        \hline
        Chi-squared         & 46.03              & \(6.85 \times 10^{-6}\) & Reject \(H_0\), accept \(H_1\)
        \\
        \hline
    \end{tabular}
\end{table}
The analysis of sleep disturbances and their impact on sleep quality yields significant findings. The Spearman correlation coefficient of -0.453 implies a moderate negative correlation, suggesting that increases in sleep disturbances are associated with deteriorations in sleep quality \cite{alhola2007sleep}. This relationship is statistically significant, with a p-value of \(4.2 \times 10^{-7}\), well below the alpha threshold of 0.05, leading us to reject the null hypothesis and affirm the alternative hypothesis \cite{morin2003cognitive}.

In addition, the Chi-squared test's statistic of 46.03, exceeding the critical value for 12 degrees of freedom, and a p-value of \(6.85 \times 10^{-6}\), strengthens the evidence against the null hypothesis. It indicates a notable association between sleep disturbances and sleep quality in our dataset, aligning with findings in existing literature \cite{riemann2014pharmacological}.

These outcomes resonate with previous research, which has highlighted various effects of sleep disturbances on sleep quality. Notably, disturbed sleep often leads to reduced overall sleep duration, falling short of the recommended 7-8 hours for adults \cite{morin2003cognitive}. Moreover, such disturbances can result in fragmented sleep, characterized by frequent awakenings and a shift away from restorative deep and REM sleep stages \cite{alhola2007sleep}. This disruption in sleep architecture is critical, as deep and REM sleep are vital for cognitive functions, emotional regulation, and overall physical health \cite{riemann2014pharmacological}.

\subsection*{Hypothesis II - Older People Have Shorter Night Sleep}
\textit{Null Hypothesis (\(H_0\))}: Age group has no impact on Sleep Duration.

\textit{Alternative Hypothesis (\(H_1\))}: Age group has a negative correlation with Sleep Duration.\\
The Pearson test was chosen for its ability to detect linear relationships between two continuous variables. Given the fairly normal distribution of Sleep Quality, it serves as an appropriate measure for this analysis. On the other hand, The Kendall’s Tau test was selected for its robustness in assessing the strength of association between two ordinal variables, making it suitable for the nature of our age group data.


\subsubsection*{Pearson Correlations Test}
The Pearson correlation coefficient was calculated as -0.232 with a p-value of 0.0088, indicating a weak negative correlation between age group and sleep duration. This suggests a trend where sleep duration may decrease slightly as age increases. Given the significance level of 0.05, the null hypothesis can be rejected in favor of the alternative.

\subsubsection*{Kendall’s Tau Test}
A Kendall’s Tau test yielded a correlation of -0.204 with a p-value of 0.00619, reinforcing the presence of a weak negative correlation. The results allow us to reject the null hypothesis, affirming the alternative that there is a statistically significant, though mild, negative association between age group and sleep duration.

\begin{table}[ht]
    \centering
    \caption{Statistical Tests for Age Group and Sleep Duration}
    \label{tab:hypothesis2}
    \begin{tabular}{|c|c|c|c|}
        \hline
        \textbf{Test Type} & \textbf{Correlation} & \textbf{P-value} & \textbf{Conclusion}            \\
        \hline
        Pearson            & -0.232               & 0.0088           & Reject \(H_0\), accept \(H_1\) \\
        \hline
        Kendall’s Tau      & -0.204               & 0.00619          & Reject \(H_0\), accept \(H_1\) \\
        \hline
    \end{tabular}
\end{table}

Both tests corroborate the alternative hypothesis, suggesting that as individuals progress into higher age categories, a decrease in sleep duration may occur. However, the correlation is not strong, pointing to the influence of other factors not accounted for in this analysis.


\subsection*{Hypothesis III - Longer Sleep Onset Time Worsens Sleep Quality}
\textit{Null Hypothesis (\(H_0\))}: Increase in Sleep Onset Time has no impact on Sleep Quality.

\textit{Alternative Hypothesis (\(H_1\))}: Increase in Sleep Onset Time leads to a decline in Sleep Quality.

The relationship between Sleep Onset Time and Sleep Quality was analyzed using Spearman's rank correlation for its suitability with ordinal data and the Chi-squared test to assess variable independence.

\begin{table}[ht]
    \centering
    \caption{Statistical Testing of Sleep Onset Time Worsens Sleep Quality}
    \label{tab:hypothesis3}
    \begin{tabular}{|c|c|c|c|}
        \hline
        \textbf{Test Type}  & \textbf{Statistic} & \textbf{P-value}        & \textbf{Conclusion}            \\
        \hline
        Spearman's \(\rho\) & -0.377641          & \(2.80 \times 10^{-5}\) & Reject \(H_0\), accept \(H_1\) \\
        \hline
        Chi-squared         & 19.297 (DoF: 9)    & 0.022782                & Reject \(H_0\), accept \(H_1\) \\
        \hline
    \end{tabular}
\end{table}

The Spearman correlation coefficient of -0.377641 suggests a moderate inverse relationship, where longer Sleep Onset Time is associated with poorer Sleep Quality. The p-value of \(2.80 \times 10^{-5}\) indicates that this correlation is statistically significant and not due to chance.

Similarly, the Chi-squared test with a statistic of 19.297119 and a p-value of 0.022782 also suggests a significant association between Sleep Onset Time and Sleep Quality. Though not a strong relationship, it is sufficient to reject \(H_0\) and accept \(H_1\), indicating a notable connection that warrants further investigation.


\subsection*{Hypothesis IVa - Weekly Exercise Frequency Improves Sleep Quality}
\textit{Null Hypothesis (\(H_0\))}: Weekly exercise frequency has no impact on Sleep Quality.

\textit{Alternative Hypothesis (\(H_1\))}: Weekly exercise frequency improves Sleep Quality.

Spearman's correlation was used to test for a potential positive association between exercise frequency and sleep quality.

\begin{table}[ht]
    \centering
    \caption{Spearman Correlation and Chi-squared Test for Exercise Frequency and Sleep Quality}
    \label{tab:hypothesis4a}
    \begin{tabular}{|c|c|c|c|}
        \hline
        \textbf{Test Type}  & \textbf{Statistic} & \textbf{P-value} & \textbf{Conclusion}            \\
        \hline
        Spearman's \(\rho\) & -0.0577            & 0.7235           & Accept \(H_0\), reject \(H_1\) \\
        \hline
        Chi-squared         & 13.2199            & 0.1529           & Accept \(H_0\), reject \(H_1\) \\
        \hline
    \end{tabular}
\end{table}

The Spearman test resulted in a correlation coefficient of -0.0577 with a p-value of 0.7235, indicating no meaningful relationship between exercise frequency and sleep quality. The negative coefficient suggests a slight inverse trend, but the high p-value shows this is likely due to chance.

The Chi-squared test further supports the lack of evidence for an association. With a Chi2 Stat of 13.2199 and a p-value of 0.1529, the result does not exceed the critical value required to reject the null hypothesis at the 0.05 significance level.

\textbf{Conclusion:}The data does not support a significant relationship between weekly exercise frequency and sleep quality. Both Spearman's correlation and the Chi-squared test outcomes suggest that other unexplored factors might be more influential in determining sleep quality.

\subsection*{Hypothesis IVb - Weekly Exercise Frequency Improves Sleep Duration}
\textit{Null Hypothesis (\(H_0\))}: Weekly exercise frequency has no impact on Sleep Duration.

\textit{Alternative Hypothesis (\(H_1\))}: Weekly exercise frequency improves Sleep Duration.

Spearman's correlation and ANOVA tests were conducted to evaluate the relationship between exercise frequency and sleep duration.

\begin{table}[ht]
    \centering
    \caption{Statistical Tests for Exercise Frequency and Sleep Duration}
    \label{tab:hypothesis4b}
    \begin{tabular}{|c|c|c|}
        \hline
        \textbf{Test Type}  & \textbf{Statistic} & \textbf{P-value} \\
        \hline
        Spearman's \(\rho\) & 0.1805             & 0.0327           \\
        \hline
        ANOVA               & 2.9423             & 0.0367           \\
        \hline
    \end{tabular}
\end{table}

The Spearman correlation coefficient of 0.1805 indicates a positive but weak relationship between exercise frequency and sleep duration, with a p-value of 0.0327 suggesting statistical significance.

The ANOVA test yields an F-statistic of 2.9423 and a p-value of 0.0367, confirming significant differences in sleep duration among different levels of exercise frequency.

\textbf{Conclusion:}
The analysis suggests that regular exercise potentially contributes to increased sleep duration, although the correlation identified is relatively weak. The ANOVA test indicates significant variations in sleep duration across different exercise frequency groups. This points to the likelihood that incorporating exercise into daily routines could lead to extended sleeping hours. However, the precise nature and extent of this relationship between exercise intensity and sleep duration warrant further investigation. Past studies have underscored the benefits of moderate-intensity exercise for enhancing sleep quality and duration, advising against intense physical activities close to bedtime due to their potential to disrupt sleep onset \cite{sleepfoundation, aasmsleep, mayoclinicsleep}.

\subsection*{Hypothesis V - Increased Nap Time Decreases Night Sleep Duration}
\textit{Null Hypothesis (\(H_0\))}: Increase in Nap Time has no impact on Night Sleep Duration.

\textit{Alternative Hypothesis (\(H_1\))}: Increase in Nap Time shortens Night Sleep Duration.

Spearman's correlation and Kendall’s Tau tests were conducted to assess the relationship between nap duration and night sleep duration.

\begin{table}[ht]
    \centering
    \caption{Statistical Tests for Nap Duration and Night Sleep Duration}
    \label{tab:hypothesis5}
    \begin{tabular}{|c|c|c|c|}
        \hline
        \textbf{Test Type}  & \textbf{Statistic} & \textbf{P-value} & \textbf{Conclusion}            \\
        \hline
        Spearman's \(\rho\) & -0.1208            & 0.1099           & Accept \(H_0\), reject \(H_1\) \\
        \hline
        Kendall’s Tau       & -0.096             & 0.1145           & Accept \(H_0\), reject \(H_1\) \\
        \hline
    \end{tabular}
\end{table}

The Spearman correlation coefficient of -0.1208 and a p-value of 0.1099 suggest a very weak negative relationship between nap duration and night sleep duration, but this is not statistically significant. Similarly, Kendall's Tau test shows a correlation coefficient of -0.096 and a p-value of 0.1145, also indicating a very weak negative correlation that does not provide enough statistical evidence to reject the null hypothesis. These results suggest that increased nap duration does not have a significant impact on reducing night sleep duration.

\subsection*{Conclusion of Hypothesis Testing}
Our statistical analysis at a significance level of 0.05 has led to several noteworthy findings regarding the impact of various factors on sleep quality and duration:

\begin{itemize}
    \item \textbf{Sleep Disturbance:} There is a significant negative correlation between sleep disturbances and sleep quality. This indicates that individuals with more sleep disruptions tend to have poorer sleep quality, underscoring the adverse effects of these disturbances.
    \item \textbf{Age Group and Sleep Duration:} A weak, yet statistically significant negative correlation exists between age group and sleep duration. This trend suggests that sleep duration may decrease slightly as age increases, highlighting the need for focused sleep health in older age groups.
    \item \textbf{Sleep Onset Time:} The analysis shows a significant relationship between sleep onset time and sleep quality. A quicker sleep onset is associated with better sleep quality, implying that faster sleep initiation is beneficial for overall sleep experience.
    \item \textbf{Exercise Weekly Frequency:} Our findings do not provide strong statistical evidence to suggest that increased exercise frequency significantly improves sleep quality. However, there is an observed correlation between higher exercise frequency and longer sleep duration, though its impact on sleep quality remains inconclusive.
    \item \textbf{Nap Duration:} The data does not establish a significant link between nap duration and night sleep duration. Therefore, it remains unclear whether napping habits substantially affect the total duration of nocturnal sleep, indicating an area for further research.
\end{itemize}

In conclusion, while certain factors like sleep disturbances and sleep onset time show clear correlations with sleep quality, other factors such as exercise frequency and nap duration present more complex relationships that require further exploration. The results underscore the multifaceted nature of sleep and its susceptibility to a range of influences, both clear and subtle.

\section{Modeling}

\subsection*{Overview}
In this modeling section, we focus on using variables that exhibited high correlations in our hypothesis testing, namely \textbf{Nap Duration}, \textbf{Exercise Days/Week}, \textbf{Sleep Disturbances}, and \textbf{Age Group}. These variables, identified as significant predictors of sleep patterns, are employed in various machine learning models to predict night sleep duration and sleep quality. The subsequent subsections detail the performance of each model, providing an analysis and comparison of their effectiveness in these prediction tasks.


\subsection{Predicting Night Sleep Duration}
The task of predicting night sleep duration was approached with multiple regression models. The models' performances were evaluated based on Mean Squared Error (MSE) and R-squared (R\(^2\)) scores.

\begin{table}[ht]
    \centering
    \caption{Regression Models: Performance Metrics for Predicting Night Sleep Duration}
    \label{tab:regression-models}
    \begin{tabular}{|l|c|c|}
        \hline
        \textbf{Model}          & \textbf{Mean Squared Error} & \textbf{R\(^2\) Score} \\
        \hline
        Linear Regression       & 1.3389                      & 0.2457                 \\
        Decision Tree Regressor & 3.4192                      & -0.9263                \\
        K-Nearest Neighbors     & 1.6241                      & 0.0850                 \\
        Support Vector Machine  & 1.5909                      & 0.1037                 \\
        \hline
    \end{tabular}
\end{table}

The Linear Regression model showed the best performance with the lowest MSE, indicating higher predictive accuracy, and a modest R\(^2\) score, suggesting a reasonable fit to the data. In contrast, the Decision Tree Regressor performed poorly, with a high MSE and a negative R\(^2\) score, indicating overfitting. KNN and SVM had similar performance levels, but both were less effective than Linear Regression.

\subsection{Predicting Sleep Quality}
For sleep quality, classification models including Multiclass Logistic Regression, Decision Tree Classifier, Random Forest, KNN, and SVM were evaluated based on accuracy and weighted average F1-score.

\begin{table}[ht]
    \centering
    \caption{Classification Models: Performance Metrics for Predicting Sleep Quality}
    \label{tab:classification-models}
    \begin{tabular}{|l|c|c|}
        \hline
        \textbf{Model}                 & \textbf{Accuracy} & \textbf{Weighted Avg F1-Score} \\
        \hline
        Multiclass Logistic Regression & 0.32              & 0.28                           \\
        Decision Tree Classifier       & 0.55              & 0.62                           \\
        Random Forest                  & 0.55              & 0.62                           \\
        K-Nearest Neighbors            & 0.50              & 0.47                           \\
        SVM                            & 0.59              & 0.52                           \\
        \hline
    \end{tabular}
\end{table}

In predicting sleep quality, the SVM model emerged as the most accurate with the highest overall accuracy and F1-score. Both Decision Tree and Random Forest models showed commendable performance but were slightly less effective than SVM. The Multiclass Logistic Regression model, however, had the lowest accuracy and F1-score, indicating its limited suitability for this particular task.

\subsection*{Modeling Summary}
The modeling analysis indicates that for predicting night sleep duration, Linear Regression provides the most reliable results, while for sleep quality, SVM is the most effective. Although some models like the Decision Tree Regressor and Multiclass Logistic Regression showed limitations in predictive power, they still contribute valuable insights. Overall, each model's performance highlights the complexity of predicting sleep-related outcomes and underscores the potential for enhanced model development with more comprehensive data.

\section{Conclusion}

Our study investigated the complex relationship between lifestyle factors and sleep, revealing key insights. Hypothesis testing and correlation analysis identified critical links: increased sleep disturbances and advancing age negatively impact sleep quality and duration, while exercise frequency showed a positive correlation with sleep duration. Nap duration, however, did not significantly affect night sleep duration.

In the modeling phase, Linear Regression effectively predicted night sleep duration, and SVM stood out in predicting sleep quality. These results highlight the nuances in sleep dynamics and the potential of machine learning to address them. Despite some limitations, our study contributes important empirical findings, suggesting that addressing specific lifestyle factors could improve sleep quality and duration. This research paves the way for future studies to further explore and refine these relationships, enhancing our understanding of sleep health.


However, our study faced several limitations:
\begin{itemize}
    \item The survey distribution mainly among friends and colleagues led to imbalanced data, potentially skewing the representation of broader populations.
    \item Time constraints limited our ability to fine-tune the models for optimal performance, which might have yielded more nuanced insights.
    \item Our limited expertise in the biological aspects of sleep may have constrained our ability to fully interpret the complex interactions between various factors and sleep patterns.
\end{itemize}

In conclusion, while providing valuable insights and contributing to the body of knowledge on sleep health, our study also paves the way for future research. It highlights the multifaceted influences on sleep and the importance of sophisticated analytical tools in understanding these complexities. The limitations noted herein can serve as guideposts for future studies, aiming to enhance our understanding and ultimately improve sleep health and overall well-being.


\subsection*{Future Work}

The insights gained from our current study pave the way for several enhancements in future research endeavors. To further refine our understanding of sleep dynamics and improve predictive modeling, we propose the following areas of focus:

\subsubsection*{Refinement of Survey Methodology}
Future iterations of the survey should involve collaboration with health professionals and reference established studies in health and sleep research. This approach will help in identifying new and important variables that could offer deeper insights into sleep patterns and their influencing factors. By grounding the survey in a more scientifically robust framework, we can ensure that it captures a comprehensive range of factors relevant to sleep quality and duration.

\subsubsection*{Addressing Data Imbalance}
One of the key challenges in our study was the data imbalance resulting from the limited scope of survey distribution. To mitigate this in future research, the survey should be disseminated across a broader demographic. This includes reaching participants beyond the authors' immediate social circles to achieve a more representative sample. Efforts should be made to ensure class balance in the dataset, which would enhance the reliability and generalizability of the findings.

\subsubsection*{Advancements in Modeling Techniques}
Given additional time and resources, fine-tuning the current models could significantly enhance their predictive capabilities. This process would involve more rigorous testing, validation, and optimization of parameters for models like Linear Regression and SVM, which showed promise in our study. Furthermore, exploring a diverse range of modeling techniques, including clustering algorithms and neural networks, could provide new perspectives and potentially more accurate predictions. These advanced methods could uncover complex patterns in the data that simpler models might miss, offering a richer understanding of the factors influencing sleep quality and duration.

In essence, the future work in this field should aim at not only addressing the limitations identified in our current research but also expanding the horizons of sleep studies through innovative methodologies and advanced analytical techniques. By doing so, we can contribute more effectively to the body of knowledge on sleep health and its critical role in overall well-being.


\subsection*{Acknowledgment}

We would like to express our deepest gratitude to a number of individuals and organizations whose support was invaluable to the success of this project:

\begin{enumerate}
    \item Our heartfelt thanks go to our families and friends. Their unwavering emotional support and active participation in our survey played a crucial role in this research. Their encouragement and involvement were key to the study's success.

    \item We extend our sincere appreciation to Professor Eunhui Kim, Yuna Jeong, and Teaching Assistant Nilesh Kumar Srivastava at KISTI. Their profound expertise, guidance in analytical techniques, and consistent support throughout the semester have greatly enriched our learning experience and contributed immensely to our research.

    \item  We express our profound appreciation to the esteemed members of our research team (Table \ref{tab:teamroles}). Their unwavering dedication, manifested through collaborative brainstorming sessions, meticulous survey design, rigorous data analysis, and the formulation of well-founded hypotheses, served as the bedrock upon which this project's success was ultimately constructed.
    \begin{table}[ht]
        \centering
        \caption{Team members and roles}
        \label{tab:teamroles}
        \begin{tabular}{|l|c|c|}
            \hline
            \textbf{Member} & \textbf{Role} \\ \hline
            Solomon Rukundo & Hypothesis testing, Presentation. \\ \hline
            Md Mahmdul Hassan  & Data analysis, visualisation, evaluation. \\ \hline
            Hoang Nguyen  & Data collection and cleaning, modelling. \\
            \hline
        \end{tabular}
    \end{table}

    \item We are grateful to the University of Science and Technology (UST) and the Korea Institute of Science and Technology Information (KISTI) for organizing this remarkable program. Their scholarship has provided us with a unique opportunity to learn and experience the technological advancements in Korea. Their support has been pivotal in our academic and professional growth.
\end{enumerate}

This project stands as a testament to the power of collaboration, mentorship, and support, and we are thankful to everyone who contributed to its success.


\bibliography{references}

\end{document}
